\begin{verbatim}
/*****************************************************************************
 * Solves a set of polynomial equations.                                     *
******************************************************************************
* (C) Gershon Elber, Technion, Israel Institute of Technology                *
******************************************************************************
* Written by:  Gershon Elber                                Ver 1.0, June 2003   *
*****************************************************************************/

/* Use exmaple:  'msolve -d 1 -c "0,1,1,0,  0,1,-1,0"' */

#include "irit_sm.h"
#include "iritprsr.h"
#include "allocate.h"
#include "attribut.h"
#include "cagd_lib.h"
#include "geom_lib.h"
#include "mvar_lib.h"

#define SUBDIV_TOL         1e-3
#define NUMERIC_TOL         1e-8
#define MIN_DOMAIN        -2
#define MAX_DOMAIN         2

static char *CtrlStr =
    "MSlove n%-NumEqns!d v%-NumVars!d d%-MaxDeg!d c%-Coeffs!s h%-";

void main(int argc, char **argv)
{
  int i, j, Error, *Lengths,
        NumOfEqnsFlag = FALSE,
        NumOfEqns = 2,
        NumOfVarsFlag = FALSE,
        NumOfVars = 2,
        MaxDegreeFlag = FALSE,
        MaxDegree = 2,
        CoefFlag = FALSE,
        HelpFlag = FALSE;
    char *Coef,
        /*        (x-1)^2 + y^2 = 2,      (x+1)^2 + y^2 = 2. */
        *Coefs = "-1,-2,1,0,0,0,1,0,0,  -1,2,1,0,0,0,1,0,0";
    MvarMVStruct **MVs, *MVTmp;
    MvarPtStruct *Pts, *Pt;
    MvarConstraintType *Constraints;

    Coefs = IritStrdup(Coefs);                /* Do not process static data. */

    if ((Error = GAGetArgs(argc, argv, CtrlStr,
                           &NumOfEqnsFlag, &NumOfEqns,
                           &NumOfVarsFlag, &NumOfVars,
                           &MaxDegreeFlag, &MaxDegree,
                           &CoefFlag, &Coefs,
                           &HelpFlag)) != 0) {
        GAPrintErrMsg(Error);
        GAPrintHowTo(CtrlStr);
        exit(1);
    }

    if (HelpFlag) {
        GAPrintHowTo(CtrlStr);
        exit(0);
    }

    printf("Processing %d equations of max degree %d, and %d variables,\nCoefs=\"%s\"\n",
           NumOfEqns, MaxDegree, NumOfVars, Coefs);

    /* Create the polynomial constraints as multivariates. */
    MVs = (MvarMVStruct **) IritMalloc(sizeof(MvarMVStruct *) * NumOfEqns);
    Lengths = (int *) IritMalloc(sizeof(int) * NumOfVars);
    for (i = 0; i < NumOfVars; i++)
        Lengths[i] = MaxDegree + 1;
    Constraints = (MvarConstraintType *) IritMalloc(sizeof(MvarConstraintType)
                                                    * NumOfEqns);
    for (i = 0; i < NumOfEqns; i++)
        Constraints[i] = MVAR_CNSTRNT_ZERO;
    Coef = strtok(Coefs, ",");

    for (i = 0; i < NumOfEqns; i++) {
        IrtRType *p;

        MVs[i] = MvarMVNew(NumOfVars, MVAR_POWER_TYPE,
                           CAGD_PT_E1_TYPE, Lengths);
        p = MVs[i] -> Points[1];

        for (j = 0; j < pow((MaxDegree + 1), NumOfVars); j++, p++) {
            if (sscanf(Coef, "%lf", p) != 1) {
                fprintf(stderr, "Error in parsing the coefficients string\n");
                exit(1);
            }
            Coef = strtok(NULL, ",");
        }

        MvarDbg(MVs[i]);

        MVTmp = MvarCnvrtPwr2BzrMV(MVs[i]);
        MvarMVFree(MVs[i]);
        MVs[i] = MVTmp;

        MvarDbg(MVs[i]);

        /* Set the domain of the multivariate. */
        for (j = 0; j < NumOfVars; j++) {
            MVTmp = MvarMVRegionFromMV(MVs[i], MIN_DOMAIN, MAX_DOMAIN, j);
            MvarMVFree(MVs[i]);
            MVs[i] = MVTmp;
        }

        MVTmp = MvarCnvrtBzr2BspMV(MVs[i]);
        MvarMVFree(MVs[i]);
        MVs[i] = MVTmp;

        for (j = 0; j < NumOfVars; j++) {
            BspKnotAffineTransOrder2(MVs[i] -> KnotVectors[j],
                                     MVs[i] -> Orders[j],
                                     MVs[i] -> Orders[j] + MVs[i] -> Lengths[j],
                                     MIN_DOMAIN, MAX_DOMAIN);
        }

        MvarDbg(MVs[i]);
    }

    Pts = MvarMVsZeros(MVs, Constraints, NumOfEqns,
                       SUBDIV_TOL, NUMERIC_TOL);
    for (Pt = Pts; Pt != NULL; Pt = Pt -> Pnext) {
        printf("(");
        for (i = 1; i <= NumOfVars; i++)
            printf("%s%.13lg", i > 1 ? ", " : "", Pt -> Pt[i-1]);
        printf(")\n");
    }  

    MvarPtFreeList(Pts);
    for (i = 0; i < NumOfEqns; i++)
        MvarMVFree(MVs[i]);
    IritFree((VoidPtr) MVs);
    IritFree((VoidPtr) Constraints);
    IritFree((VoidPtr) Lengths);

    exit(0);
}
\end{verbatim}
